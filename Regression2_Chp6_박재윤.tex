\documentclass{beamer}
%
% Choose how your presentation looks.
%
% For more themes, color themes and font themes, see:
% http://deic.uab.es/~iblanes/beamer_gallery/index_by_theme.html
%
\mode<presentation>
{
	\usetheme{default}      % or try Darmstadt, Madrid, Warsaw, ...
	\usecolortheme{default} % or try albatross, beaver, crane, ...
	\usefonttheme{default}  % or try serif, structurebold, ...
	\setbeamertemplate{navigation symbols}{}
	\setbeamertemplate{caption}[numbered]
	\setbeamertemplate{footline}[frame number]
} 

% \usepackage{kotex} 

\usepackage{times}
\usepackage{amsmath}
\usepackage{multirow}

\newenvironment{rcases}
{\left.\begin{aligned}}
	{\end{aligned}\right\rbrace}


\usepackage{color}
\pagecolor{yellow}
\title{ {\Large \bf Regression Analysis II }  }  
\subtitle{Chapter 6. Transformation of the Linear Regression Model}
\author{Hojin Yang}
\institute{Department of Statistics\\ Pusan National University}
\date{}



\begin{document}
	\scriptsize
	
	
	\begin{frame}
		\titlepage
	\end{frame}
	
	% Uncomment these lines for an automatically generated outline.
	%\begin{frame}{Outline}
	%  \ofcontents
	%\end{frame}
	
	
	%%%%%%%%%%%%%%%%%%%%%%%%%%%%%%%%%%%%%%%%%%%%%%%%%%%%%%%%%%%%%%%%%%%%%%%%%%
	
	\frame{ 
		\frametitle{6.1 The Use of Dummy Variable}
		\textbf{6.1.1 One dummy variable case} \\
		\vspace{1em}
		(1) Motivating example \\
		\vspace{0.5em}
		Problem : A company makes advertisements either in the newspaper or the broadcast. \\
		\vspace{0.5em}
		They want to see the difference in the advertisement effect between the newspaper \\
		\vspace{0.5em}
		and the broadcast. \\
		\vspace{0.5em}
		Solution : Let
		\\
		$$
		Z =
		\begin{cases}
			1, & \mbox{if newspaper} \\
			0, & \mbox{if broadcast}
		\end{cases}
		$$
		and consider
		$$
		Y = \beta_0 + \beta_1X + \beta_2Z + \varepsilon,
		$$
		\\
		where $Z$ is called \emph{dummy variable(indicator variable)}. Hence,
		$$ \begin{aligned}
			& Z = 0 : E[Y] = \beta_0 + \beta_1X \\
			& Z = 1 : E[Y] = (\beta_0 + \beta_2) + \beta_1X
		\end{aligned}
		$$
		\\
		Test : $H_0 : \beta_2$ = 0 
	}
	
	%%%%%%%%%%%%%%%%%%%%%%%%%%%%%%%%%%%%%%%%%%%%%%%%%%%%%%%%%%%%%%%%%%%%%%%%%%
	
	\frame{ 
		\frametitle{6.1 The Use of Dummy Variable}
		\textbf{Remark} \\
		\vspace{0.5em}
		If a categorical variable has $c$ categories, then we must have $c-1$ dummy variables. \\
		\vspace{0.5em}
		For example, in this problem, we have two categories, so that we need 1( = $2-1$) \\  
		\vspace{0.5em}
		dummy variable. Now, assume that we defined two dummy variables in this example, \\
		\vspace{0.5em}
		i.e., 
		\\
		$$ 
		Z_1  =
		\begin{cases}
			1, & \mbox{if newspaper} \\
			0, & \mbox{if otherwise}
		\end{cases} 
		, \; Z_2  =
		\begin{cases}
			1, & \mbox{if broadcast} \\
			0, & \mbox{if otherwise}
		\end{cases} 
		$$
		\\
		Then, the corresponding model would be
		\\
		$$
		Y = \beta_0 + \beta_1X + \beta_2Z_1 + \beta_3Z_2 + \varepsilon
		$$
		\\
	}
	
	%%%%%%%%%%%%%%%%%%%%%%%%%%%%%%%%%%%%%%%%%%%%%%%%%%%%%%%%%%%%%%%%%%%%%%%%%%
	
	\frame{ 
		\frametitle{6.1 The Use of Dummy Variable}
		then the design matrix might be, for example,
		$$
		\textbf{X} = 
		\begin{bmatrix}
			1 & X_1 & 1 & 0 \\
			1 & X_2 & 1 & 0 \\
			1 & X_3 & 1 & 0 \\
			1 & X_4 & 1 & 0 \\
			1 & X_5 & 1 & 0 \\
			1 & X_6 & 0 & 1 \\
			1 & X_7 & 0 & 1 \\
			1 & X_8 & 0 & 1 \\
			1 & X_9 & 0 & 1 \\
			1 & X_{10} & 0 & 1 \\
		\end{bmatrix}
		$$
		which is not a full rank matrix because the sum of the 3rd and 4th columns become the \\
		\vspace{0.5em}
		1st column. Therefore, $\textbf{X}'\textbf{X}$ is not invertible.
	}
	
	%%%%%%%%%%%%%%%%%%%%%%%%%%%%%%%%%%%%%%%%%%%%%%%%%%%%%%%%%%%%%%%%%%%%%%%%%%
	
	\frame{ 
		\frametitle{6.1 The Use of Dummy Variable}
		(2) Example \\
		\vspace{0.5em}
		We have 4 types of seed ; A, B, C, and D, and want to see the difference in products between seeds. Since this categorical variable has 4 categories, we must define 3( = $4-1$) dummy variables. Hence, define
		$$
		Z_1  =
		\begin{cases}
			1, & \mbox{if A} \\
			0, & \mbox{if otherwise}
		\end{cases} 
		, \; Z_2  =
		\begin{cases}
			1, & \mbox{if B} \\
			0, & \mbox{if otherwise}
		\end{cases} 
		, \; Z_3  =
		\begin{cases}
			1, & \mbox{if C} \\
			0, & \mbox{if otherwise}
		\end{cases} 
		$$
		Note that seed D is given when $Z_1$ = $Z_2$ = $Z_3$ = 0. Hence, the model becomes
		$$
		Y = \beta_0 + \beta_1Z_1 + \beta_2Z_2 + \beta_3Z_3 + \varepsilon
		$$
		and the expectation of product of each seed is
		$$
		\begin{aligned}
			& A : E[Y] = \beta_0 + \beta_1 \\
			& B : E[Y] = \beta_0 + \beta_2 \\
			& C : E[Y] = \beta_0 + \beta_3 \\
			& D : E[Y] = \beta_0
		\end{aligned}
		$$
		For example, the difference of products between A and B is $\beta_2 - \beta_1$
	}
	
	%%%%%%%%%%%%%%%%%%%%%%%%%%%%%%%%%%%%%%%%%%%%%%%%%%%%%%%%%%%%%%%%%%%%%%%%%%
	
	\frame{ 
		\frametitle{6.1 The Use of Dummy Variable}
		\textbf{6.1.2 Extension of the use of dummy variables} \\
		\vspace{1em}
		(1) Interaction term \\
		\vspace{0.5em}
		If we include the interaction term, then
		$$
		Y = \beta_0 + \beta_1X + \beta_2Z + \beta_3XZ + \varepsilon
		$$
		so that
		$$ \begin{aligned}
			& Z = 0 : E[Y] = \beta_0 + \beta_1X \\
			& Z = 1 : E[Y] = (\beta_0 + \beta_2) + (\beta_1 + \beta_3)X
		\end{aligned}
		$$
		Here, both the slope and the intercept are different. \\
		\vspace{1em}
		(2) Break(change) point \\
		\vspace{0.5em}
		If we have a prior information on the change point, for example, 
		$$
		Z =
		\begin{cases}
			1, & \mbox{X $>$ 20} \\
			0, & \mbox{X $\leq$ 20}
		\end{cases}
		$$
		
	}
	
	%%%%%%%%%%%%%%%%%%%%%%%%%%%%%%%%%%%%%%%%%%%%%%%%%%%%%%%%%%%%%%%%%%%%%%%%%%
	
	\frame{ 
		\frametitle{6.1 The Use of Dummy Variable}
		then, we can model like
		\\
		$$
		Y = \beta_0 + \beta_1X + \beta_2(X-20)Z + \varepsilon
		$$
		\\
		so that
		\\
		$$\begin{aligned}
			& 1. X \leq 20 : E[Y] = \beta_0 + \beta_1X \\
			& 2. X > 20 : E[Y] = (\beta_0 - 20\beta_2) + (\beta_1 + \beta_2)X
		\end{aligned}
		$$
		\\
		\vspace{2em}
		\textbf{6.1.3 ANOVA using the regression model} \\
		\vspace{1em}
		Consider the following one-way ANOVA model
		$$
		Y_{ij} = \mu + \alpha_i + \varepsilon_{ij}, \quad i = 1, 2, \cdots , c \; ; \; j = 1, 2, \cdots, n_i,
		$$
		where $Y_{ij}$ is the $j$-th response of the $i$-th group, $\mu$ is a overall effect, $\alpha_i$ is the effect of \\
		\vspace{0.5em}
		the $i$-th group, $\varepsilon_{ij}$ is error term.
	}
	
	%%%%%%%%%%%%%%%%%%%%%%%%%%%%%%%%%%%%%%%%%%%%%%%%%%%%%%%%%%%%%%%%%%%%%%%%%%
	
	\frame{ 
		\frametitle{6.1 The Use of Dummy Variable}
		 For $c$ = 4, and $n_i$ = 3, we have
		$$ \begin{aligned}
			& \boldsymbol{\beta} = (\mu, \alpha_1, \alpha_2, \alpha_3, \alpha_4)' \\
			& \textbf{X} = 
			\begin{bmatrix}
				1 & 1 & 0 & 0 & 0 \\
				1 & 1 & 0 & 0 & 0 \\
				1 & 1 & 0 & 0 & 0 \\
				1 & 0 & 1 & 0 & 0 \\
				1 & 0 & 1 & 0 & 0 \\
				1 & 0 & 1 & 0 & 0 \\
				1 & 0 & 0 & 1 & 0 \\
				1 & 0 & 0 & 1 & 0 \\
				1 & 0 & 0 & 1 & 0 \\
				1 & 0 & 0 & 0 & 1 \\
				1 & 0 & 0 & 0 & 1 \\
				1 & 0 & 0 & 0 & 1 
			\end{bmatrix}
		\end{aligned}
		$$
	}
	
	%%%%%%%%%%%%%%%%%%%%%%%%%%%%%%%%%%%%%%%%%%%%%%%%%%%%%%%%%%%%%%%%%%%%%%%%%%
	
	\frame{ 
		\frametitle{6.1 The Use of Dummy Variable}
		But, $\textbf{X}$ is not a full rank matrix, so that it is not estimable, To overcome this difficulty, \\
		\vspace{0.5em}
		use dummy variables. Let
		\\
		$$
		Z_1 =
		\begin{cases}
			1, & \mbox{group 1} \\
			0, & \mbox{otherwise}
		\end{cases}
		, \; Z_2 =
		\begin{cases}
			1, & \mbox{group 2} \\
			0, & \mbox{otherwise}
		\end{cases}	
		, \; Z_3 = 
		\begin{cases}
			1, & \mbox{group 3} \\
			0, & \mbox{otherwise}
		\end{cases}
		$$
		\\
		Then,
		\\
		$$
		Y_i = \beta_0 + \beta_1Z_{i1} + \beta_2Z_{i2} + \beta_3Z_{i3} + \varepsilon_i, \quad i = 1, 2, \cdots, 12
		$$
		\\
		Note that the test for $H_0$ : $\alpha_1 = \alpha_2 = \alpha_3 = \alpha_4 = 0$ is equivalent to \\
		\vspace{0.5em}
		$H_0$ : $\beta_1 = \beta_2 = \beta_3 = 0$ \\
		\vspace{2em}
		Ex. 6.3 (p.242)
	}
	
	%%%%%%%%%%%%%%%%%%%%%%%%%%%%%%%%%%%%%%%%%%%%%%%%%%%%%%%%%%%%%%%%%%%%%%%%%%
	
	\frame{ 
		\frametitle{6.2 Polynomial Regression}
		1st order polynomial regression \\
		\vspace{1.5em}
		\quad 1 covariate : $Y = \beta_0 + \beta_1X + \varepsilon$ \\
		\vspace{1em}
		\quad 2 covariates : $Y = \beta_0 + \beta_1X_1 + \beta_2X_2 + \varepsilon$ \\
		\vspace{1em}
		\quad $k$ covariates : $Y = \beta_0 + \sum_{i=1}^{k} \beta_iX_i + \varepsilon$ \\
		\vspace{2.5em}
		2nd order polynomial regression \\
		\vspace{1.5em}
		\quad 1 covariate : $Y = \beta_0 + \beta_1X + \beta_2X^2 + \varepsilon$ \\
		\vspace{1em}
		\quad 2 covariates : $Y = \beta_0 + \beta_1X_1 + \beta_2X_2 + \beta_{11}X^2_1 + \beta_{22}X^2_2 + \beta_{12}X_1X_2 + \varepsilon$ \\
		\vspace{1em}
		\quad $k$ covariates : $Y = \beta_0 + \sum_{i=1}^{k} \beta_iX_i + \sum\beta_{ii}X^2_i + \sum_{i<j}^{k} \beta_{ij}X_iX_j + \varepsilon$
	}

	

	
	
	
	
	
	
	
	
	
\end{document}